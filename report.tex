\documentclass[a4paper,11pt]{article}

\usepackage{url}
\usepackage{graphicx}
\usepackage{enumerate}
\usepackage{amsmath}
\usepackage{float}
\usepackage{longtable}
%\usepackage{fullpage}
\usepackage{pstricks}
\usepackage{tikz}
\usepackage[absolute]{textpos}
\usepackage{import}
\usepackage{subfigure}
\usepackage{setspace}

\title{G54MDP Individual Report}
\author{Marcus Whybrow (mxw18u)} \date{\today}

% Dutch style paragraph formatting
\setlength{\parskip}{1.3ex plus 0.2ex minus 0.2ex}
\setlength{\parindent}{0pt}

%\doublespacing
\onehalfspacing

\begin{document}
    \maketitle
    
    \section{Introduction}
    
    The application which we have created is written for the iOS platform and aims to offer the unified inbox feature of the stock mail app in the guise of an RSS reader.
    
    Choosing a type of app to develop was not easy, and the four of us - Rob Golding, Rob Miles, Michal Konturek and myself - initially were in contention between developing a multi-device multiplayer pong style game and a more standard concept such as an RSS reader.
    
    Additionally there was much debate around the chosen platform, swinging from Android to iOS, back to Android and then finally back to iOS where it was solidified for before development began.
    
    Once we had decided upon creating a ``get out of your way" RSS reader application, the animation and user interaction paradigms available (and out in the wild) within iOS apps offered too much of a benefit over Android, and the decision to develop an iOS app was made.
    
    RSS feeds are a documents which websites provide, defining a list of pages within the website. Each page in the feed is represented by a list of required and optionally defined attributes such as page URL, title and description. An RSS document would typically list the 10 newest pages added to a website, allowing RSS readers to request the document and obtain a list of links and descriptions of the 10 latest items.
    
	Our application entitled ``FeedReader" aims to allow users to keep up to date with the various RSS feeds which they follow by allowing users to track RSS feeds. For each feed, it keeps track of which items have already been read, and allows old items to be deleted once finished with. Each tracked feed can be refreshed - which requests the respective RSS document again - allowing any new items found to be added to the app.
    
    \section{Pulse}
    
    \section{Technologies}
    
    \section{Architecture}
    
    \section{The Group}
    
    \section{Personal Contribution}
    
    \section{Conclusion}

\end{document}
